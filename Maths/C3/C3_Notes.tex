\documentclass[a4paper]{article}

%% Language and font encodings
\usepackage[english]{babel}
\usepackage[utf8x]{inputenc}
\usepackage[T1]{fontenc}

%% Sets page size and margins
\usepackage[a4paper,top=3cm,bottom=2cm,left=3cm,right=3cm,marginparwidth=1.75cm]{geometry}

%% Useful packages
\usepackage{amsmath}
\usepackage{amssymb}
\usepackage{amsfonts}
\usepackage{float}
\usepackage{graphicx}
\usepackage{multicol}

\usepackage[colorinlistoftodos]{todonotes}
\usepackage[colorlinks=true, allcolors=black]{hyperref}

\usepackage{mathtools}

\makeatletter
\renewcommand\thesection{}
\renewcommand\thesubsection{\@arabic\c@section.\@arabic\c@subsection}
\renewcommand\thesubsubsection{}
\makeatother
\newcommand\inv[1]{#1\raisebox{1.15ex}{$\scriptscriptstyle-\!1$}}

\title{Core 3 Notes}
\author{Jacob Santry}

\begin{document}
\maketitle
\tableofcontents

\section{Differentiation}
\subsection{C1 Differentiation reminder}

\[y=ax^n\]
\[\Rightarrow \frac{dy}{dx}=anx^{n-1} \]
\paragraph{Remember}\mbox{}\\
$\frac{dy}{dx}$ is a measure of the rate of change of y compared to x. e.g. $\frac{dy}{dx}$=3 then y increases by 3 for every increase of 1 in x.\\
If two lines are perpendicular the the product of their gradients is -1.

\subsection{C2 Differentiation}
Chain rule ( or "function of function rule") \\
\[\frac{dy}{dx}=\frac{dy}{du}\times\frac{du}{dx}\]

\begin{multicols}{3}
\large
\paragraph{Ex1}
\[y=(3x+2)^5\]
\[\text{let }u=3x+2\]
\[\Rightarrow y=u^5 \]
\[\frac{dy}{dx}=\frac{dy}{du}\times\frac{du}{dx}\]
\[\frac{dy}{dx}=5u^4\times3\]
\[\quad = 5u^4 \times 3 \]
\[\quad = 15u^4 \]
\[\quad = 15(3x+2)^4 \]

\paragraph{Ex2}
\[y=\sqrt{4x^2-1}\]
\[\quad = (4x^2-1)^{\frac{1}{2}} \]
\[\text{let }u = 4x^2 -1 \]
\[\Rightarrow y=u^{\frac{1}{2}} \]
\[\frac{dy}{dx}=\frac{dy}{du}\times\frac{du}{dx}\]
\[\quad \frac{1}{2}u^{-\frac{1}{2}}\times 8x \]
\[\quad \frac{4x}{(4x^2-1)^{\frac{1}{2}}} \]

\paragraph{Ex3}
\[y=\frac{1}{\sqrt[3]{5-4x}}\]
\[\quad = (5-4x)^{-\frac{1}{3}} \]
\[\text{let } u = 5-4x \]
\[y=u^{-\frac{1}{3}} \]
\[\frac{dy}{dx}=\frac{dy}{du}\times\frac{du}{dx}\]
\[\quad = -\frac{1}{3}U^{\frac{-4}{3}}\times -4 \]
\[\quad = \frac{4}{3}u^{-\frac{4}{3}} \]
\[\quad = \frac{4}{3}(5-4x)^{-\frac{4}{3}} \]
\normalsize
\end{multicols}

\paragraph{General Rule}\mbox{}\\
If \[y=\big[ f(x) \big]^n\]
then \[\frac{dy}{dx} = n \big[ f(x) \big]^{n-1} \times f'(x)\]
\[y=(4x^2+9)^5\]
\[\frac{dy}{dx} = 5(4x^2+9)^4 \times 8x\]
\[\quad = 40x(4x^2+9)^4\]
\\
One particular example of our general rule is
\[\text{if }y=(ax+b)^n \]
\[\text{then }\frac{dy}{dx} = an(ax+b)^{n-1} \]
and by considering intergration as the reverse of our differentiation we cna deduce that
\large
\[\int(ax+b)^n dx = \frac{(ax+b)^{n+1}}{a(n+1)}+c \quad \text{for } n\neq-1 \]
e.g.
\begin{enumerate}
  \item
  \[\int(2x+3)^5dx=\frac{(2x+2)^6}{2x+6}+c\]
  \[\quad = \frac{(2x+3)^4}{12}+c \]
  \item
  \[\int^{2}_{-\frac{1}{3}} \frac{1}{\sqrt[3]{3x+2}}dx = \int^{2}_{-\frac{1}{3}} (3x+2)^{-\frac{1}{3}}dx \]
  \[\quad = \bigg[\frac{(3x+2)^{\frac{2}{3}}}{2} \bigg]^{2}_{-\frac{1}{3}} \]
  \[\quad = 2-\frac{1}{2} =\frac{3}{2} \]
  \normalsize
\end{enumerate}

\paragraph{The link between $\frac{dy}{dx}$ and $\frac{dx}{dy}$}\mbox{}\\
$\Big\lceil$ If $\frac{dy}{dx}$ = 4 then y is increasing 4 times as fast as x. This would mean that x is increasing $\frac{1}{4}$ times as fast as y.\\
\[\text{i.e.} \frac{dy}{dx} = \frac{1}{4} \Big\rfloor \]
Hence
\[\Large
\framebox[1.1\width]{
$\frac{dy}{dx} = \frac{1}{\frac{dy}{dx}}$
}
\normalsize\]

\begin{enumerate}
  \item (Volume of a sphere $=\frac{4}{3}\pi r^3$)\\
  The radius of a sphere is increasing at 2cm\inv{\text{s}}.\\
  Find the rate of increase of volume at the instant that the radius is 5cm.
  \item (Surface area of a sphere $=4\pi r^2$)
\end{enumerate}
\end{document}
