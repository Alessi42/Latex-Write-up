\documentclass[a4paper]{article}

%% Language and font encodings
\usepackage[english]{babel}
\usepackage[utf8x]{inputenc}
\usepackage[T1]{fontenc}

%% Sets page size and margins
\usepackage[a4paper,top=3cm,bottom=2cm,left=3cm,right=3cm,marginparwidth=1.75cm]{geometry}

%% Useful packages
\usepackage{amsmath}
\usepackage{amssymb}
\usepackage{amsfonts}
\usepackage{float}
\usepackage{graphicx}

\usepackage[colorinlistoftodos]{todonotes}
\usepackage[colorlinks=true, allcolors=blue]{hyperref}

\usepackage{mathtools}

\makeatletter
\renewcommand\thesection{}
\renewcommand\thesubsection{\@arabic\c@section.\@arabic\c@subsection}
\renewcommand\thesubsubsection{}
\makeatother
\newcommand{\mathsection}[1]{\paragraph{#1}\mbox{}\\}

\title{Further Pure 3 Notes}
\author{Jacob Santry}

\begin{document}
\maketitle
\tableofcontents

\section{Group Theory}
\subsection{Background Work}
\normalsize
\paragraph{Set Notation}\mbox{}\\
\large
\[ A=\{\text{Colours of the rainbow}\} \]
\[ Red \underset{\mathclap{\substack{\uparrow \\ \text{Is an element of.}}}}{\in} A \]
\[ \text{or } B\{2,5,11,17\}\]
\[ \text{or } C =\{x \underset{\mathclap{\substack{\uparrow \\ \text{Such that.}}}}{:} x>2 \}\]

\normalsize
\paragraph{Special sets}\mbox{}\\ \\
\large
$\mathbb{R}$=\{\text{Real numbers}\} \\
$\mathbb{C}$=\{\text{Complex numbers}\}  \\
$\mathbb{Z}$=\{\text{Integers}\} \\
$\mathbb{Z^+}$=\{\text{Real numbers}\}  \\
\[ \big[ \text{also } \mathbb{N}=\{\text{Natural numbers}\} \big] \]} \\
$\mathbb{Q}$=\{\text{Rational numbers}\}

\normalsize
\paragraph{Binary Operations}\mbox{}\\ \\
The operation, * , acts on the two elements of a set to produce a unique "product" (i.e. outcome).

\paragraph{Properties of Binary Operations}\mbox{}
\begin{enumerate}
\item
  Commutivity: an operation, *, is communative iff
  \Large
  \[ x * y = y * x \quad \forall x,y. \]
  \normalsize
\item
  Associativity: operation, *, is associative iff
  \Large
  \[ x*(y*z) = (x*y)*z \quad \forall x,y,z \]
  \normalsize
\item
  Distributive property: operation, *, is distributive over operation \circ, iff \[\]
  \Large
  \[ x*(y\circ z) = (x * y)\circ(x * z) \quad \forall x,y,z \]
  \normalsize

\subsection{Modulo Arithmetic}

Aritmetic modulo n returns the remainder after division by n. \\
e.g.
\Large
\[ 3 +_6 5 = 2 \]
\[ 3 \times_6 5 = 3 \]
\normalsize
both of these calculations are restricted to the set $\mathbb{Z}$ = \{0,1,2,3,4,5\} \\
so $\mathbb{Z}$_n = \{0,1,\dots,n-1\}

\paragraph{Definitions of a group}\mbox{}\\ \\

\end{enumerate}
\end{document}
