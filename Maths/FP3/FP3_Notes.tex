\documentclass[a4paper]{article}

%% Language and font encodings
\usepackage[english]{babel}
\usepackage[utf8x]{inputenc}
\usepackage[T1]{fontenc}

%% Sets page size and margins
\usepackage[a4paper,top=3cm,bottom=2cm,left=3cm,right=3cm,marginparwidth=1.75cm]{geometry}

%% Useful packages
\usepackage{amsmath}
\usepackage{amssymb}
\usepackage{float}
\usepackage{graphicx}

\usepackage[colorinlistoftodos]{todonotes}
\usepackage[colorlinks=true, allcolors=blue]{hyperref}

\usepackage{mathtools}

\makeatletter
\renewcommand\thesection{}
\renewcommand\thesubsection{\@arabic\c@section.\@arabic\c@subsection}
\renewcommand\thesubsubsection{}
\makeatother
\newcommand{\mathsection}[1]{\paragraph{#1}\mbox{}\\}

\title{Further Pure 3 Notes}
\author{Jacob Santry}

\begin{document}
\maketitle
\tableofcontents

\section{Group Theory}
\subsection{Background Work}
\normalsize
\paragraph{Set Notation}\mbox{}\\
\Large
\[ A=\{\text{Colours of the rainbow}\} \]
\[ Red \underset{\mathclap{\substack{\uparrow \\ \text{Is an element of.}}}}{∈} A \]
\[ \text{or } B\{2,5,11,17\}\]
\[ \text{or } C =\{x \underset{\mathclap{\substack{\uparrow \\ \text{Such that.}}}}{:} x>2 \}\]

\normalsize
\paragraph{Special sets}\mbox{}\\ \\
\Linebreak
\large
\mathbb{R}=\{\text{Real numbers}\} \\
\mathbb{C}=\{\text{Complex numbers}\}  \\
\mathbb{Z}=\{\text{Integers}\} \\
\mathbb{Z^+}=\{\text{Real numbers}\} \\
\[ \big[ \text{also } \mathbb{N}=\{\text{Natural numbers}\} \big] \]} \\
\mathbb{Q}=\{\text{Rational numbers}\} \\
\end{document}
